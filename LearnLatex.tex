% =============== 导言区 ===============
\documentclass{ctexart}

% AMS数学包
\usepackage{amsmath}
\usepackage{amsfonts}
\usepackage{amssymb}

% XeLaTeX logo
\usepackage{metalogo}	

% 支持语法高亮的代码环境:lstlisting
\usepackage{listings}



% 采用titlepage自定义页面,以下信息可注释
%\title{\LaTeX 基础}
%\author{XiaoCY}
%\date{\today}

\newtheorem{thm}{定理}[section]

\bibliographystyle{plain}

% =============== 正文区 ===============
\begin{document}
	
%	\maketitle	% 由titlepage代替:自定义封面
	\begin{titlepage}
		\ % 产生空格占位
		
		\vspace{\fill}
		
		\begin{flushright}
			{\Huge\bfseries\LaTeX 使用简介} \\
			\rule[1em]{\linewidth}{0.5ex}
			
			\begin{tabular}{rr}
				作者 & XiaoCY \\
				版本 & 1.0 \\
				完成日期 & 2020-02-04 \\
				最后修改 & 2020-02-04
			\end{tabular}
		\end{flushright}
		
		\vspace{\stretch{3}}
		
		{\noindent Email: chunyu2018@foxmail.com}
	\end{titlepage}

	\tableofcontents
	\clearpage
	
	\section{写在前面}
	\LaTeX 是一种区别于Word的排版软件,
	其格式由各种命令、环境控制,
	很容易做到内容与格式分离,
	这是它区别于Word的重要一点。
	这篇文章主要是本人学习 \LaTeX 的记录\cite{liuhaiyang},
	记录将常用的基本操作以便后期查阅,
	同时也为感兴趣的小伙伴们做一个粗略的介绍。
	
	本人学习时安装的发行版为 \TeX~Live,		% ~为空格
	对于不知道如何入门的小伙伴,
	我同样推荐这个发行版,
	它避免了很多繁琐的配置。
	编译器以\XeLaTeX 为主,
	编辑器可采用TeXstudio。
	
	\section{基础介绍}
	\subsection{文档组成}
	\LaTeX 文档的格式通常是以后缀.tex结尾的文本文件,
	为了正常使用中文和Unicode的特殊符号,
	务必将文本以UTF-8编码进行保存。
	
	\LaTeX 文档可分为导言区和正文区。
	导言区在命令\verb|\begin{document}|之前,
	通常对文档的性质做一些设置,
	也可以自定义一些命令。
	导言区之后为正文区,
	是文档的主要内容。
	
	在导言区内,我们首先需要声明文档类,
	例如本文声明C\TeX 文档类时采用了命令\verb|\documentclass{ctexart}|。
	此外,很多优秀的宏包可以辅助我们对文档格式进行控制,
	为了使用这些宏包,
	我们需要在导言区使用命令\verb|\usepackage{<宏包名>}| 。
	(如无特殊说明,本文的命令示意中以尖括号连同内容表示命令,
	使用时不需要添加尖括号。)
	
	在正文区内,我们只需要输入文档的正文即可。
	编辑器内部的各种换行不会引起排版后文档的换行,
	中 文 文 档 的 多 余 空 格 也 不 会 在 文 档 中 出 现。 \par
	文字换行可用双斜线\verb|\\| 实现,
	而换段则采用空行的形式,或者使用命令\verb|\par|,
	.tex文件内的多个空行不会在正文引起多余的空行。
	换行和换段的区别是:
	当设置了段落首行缩进时,
	采用双斜线的换行不会引入缩进。
	
	\subsection{设置字体}
	
	\subsection{特殊字符}
	\subsection{定义命令}
	\LaTeX 中的命令又称为宏,它们都以反斜线开头,
	一般格式为:
	
	\begin{quote}	% 引用环境:增加前后间距和缩进
		\begin{tabular}{ll}
			无参数: & \verb|\command| \\
			有n个参数: & \verb|\command{<arg1>}{<arg2>}...{<argn>}| \\
			有可选参数: & \verb|\command[<opt>]{<arg1>}{<arg2>}...{<argn>}| 
		\end{tabular}
	\end{quote}

	例如,本文档在生成目录时就采用了无参数的命令\verb|\tableofcontents|。
		
	\subsection{环境介绍}
	与命令相似,环境也可分为有参数的环境和无参数的环境。
	有参数环境的一般格式为:
	
	\begin{quote}
		\verb|\begin{<环境名>}[<可选参数>]{<其他必要参数>}|  \\
		\verb|<环境内容>|	\\
		\verb|\end{<环境名>}| 
	\end{quote}
	
	一个环境就是一个分组,它限定了一些命令的作用范围。
	除了使用环境,也可以用成对的花括号\{\}直接产生一个分组。
	
	这里我们举个例子说明分组的作用:
	比如现在想修改部分字体设置为楷体,
	这可以通过无参数的命令\verb|\kaishu|来实现。
	但是直接使用该命令会导致以后的文字全部是楷体,
	这时只需要\verb|{\kaishu 像这样}|构造分组,
	就能得到{\kaishu 像这样}的局部楷体内容,
	而不会引起后面字体的改变。
	
	引文环境quote是一种无参数环境,常用来引用大段文字。
	它将环境中的内容单独分行,
	增加缩进和上下间距排印,
	以突出引用的部分。
	前面环境一般格式的介绍即采用了这种环境。
	
	定理环境是一类环境,在使用前需要在导言区进行定义:
	
	\verb|\newtheorem{thm}{定理}|
	
	这样我们就得到了一个thm环境,
	它在使用时会自动产生形如“定理1”的提示。
	用同样的方法我们还可以定义引理、公理等。
	有时候我们希望定理的编号包含章节号,
	可在定义时增加参数即可:
	
	\verb|\newtheorem{thm}{定理}[section]|
	
	定理环境还可以有一个可选参数,即定理的名字。
	下面给出这个环境的使用示例:
	
	\begin{thm}[勾股定理]
		直角三角形斜边的平方等于两直角边的平方和。
	\end{thm}
	
	表格环境table和图片环境figure均为浮动环境。
	与一般的环境不同,
	浮动环境可以根据文档的内容改变位置。
	在使用浮动环境时可利用可选参数h、t、b、p的组合来指明允许的浮动位置。
	其中:h表示代码中与前后文关系不变的位置;
	t表示当前页面的顶端;
	b表示当前页面的底端;
	p表示下一页。
	
	
	
	\section{编写文档}
	\subsection{段落间距}
	\subsection{内容强调}
	\subsection{添加列表}
	\subsection{绘制表格}
	\subsection{插入图片}
	\subsection{数学公式}
	\subsection{插入代码}
	\subsection{插入引用}
	\subsection{使用颜色}
	\subsection{页面控制}
	
	\section{使用模板}
	
	
	\bibliography{ref}
\end{document}
