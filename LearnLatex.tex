% =============== 导言区 ===============
\documentclass[hyperref,UTF8]{ctexart}

% 设置页边距的宏包
\usepackage{geometry}
\geometry{ % 设置整体页面格式
	a4paper,
	left = 2.1cm,
	right = 2.1cm,
	bottom = 2.1cm,
	top = 2.5cm
}

% 设置章节标题格式
%\CTEXsetup[
%	name = {\S\ ,},
%	aftername = {\hspace{2ex}}
%	]{section}

% AMS数学包
\usepackage{amsmath}
\usepackage{amsfonts}
\usepackage{amssymb}
\numberwithin{equation}{section}			% 设置公式编号包含节编号

% XeLaTeX logo
\usepackage{metalogo}	

% 使用颜色
\usepackage{xcolor}

% 支持语法高亮的代码环境:lstlisting
\usepackage{listings}
\lstset{ % 代码环境整体设置
	basicstyle = \ttfamily,
	keywordstyle = \bfseries,
	commentstyle = \rmfamily\upshape,
	stringstyle = \ttfamily\slshape,
	tabsize = 4,
	backgroundcolor = \color{lightgray},	% 需先引入颜色宏包
	frame = single,
	language = TeX							% 设置默认语言
}

% 优化脚注
\usepackage[perpage]{footmisc}	% 脚注每页清零
\usepackage{pifont}				% 优化带圈的数字
\renewcommand{\thefootnote}{\ding{\numexpr171+\value{footnote}}}

% 图表宏包
\usepackage{graphicx}			% 插图宏包\includegraphics
\usepackage{tabularx}			% 定宽表格

% 三线表宏包
% 顶部粗线: \toprule
% 中间细线: \midrule, \cmidrule{a-b}
% 底部粗线: \bottomrule
\usepackage{booktabs}

% 参考文献宏包
\usepackage[super,square]{natbib}

% 定义定理类
\usepackage[thmmarks]{ntheorem}
\newtheorem{thm}{定理}[section]
{	% 利用分组使设置只对该分组内的定理类有效
	\theoremstyle{nonumberplain}
	\theoremheaderfont{\bfseries}
	\theorembodyfont{\normalfont}
	\theoremsymbol{\ensuremath{\Box}}
	\newtheorem{proof}{证明}
}



% 定义四元数乘法的简写
\newcommand{\quadprod}[2]
	{\ensuremath{\mathfrak{#1} \otimes \mathfrak{#2}}}

\bibliographystyle{unsrtnat}

% 采用titlepage自定义页面,以下信息可注释
%\title{\LaTeX 基础}
%\author{XiaoCY}
%\date{\today}

% =============== 正文区 ===============
\begin{document}

%	\maketitle	% 由titlepage代替:自定义封面
	\begin{titlepage}
		\zihao{4}	% 设置封面采用四号字
		\ % 产生空格占位
		
		\vspace{\fill}
		
		\begin{flushright}
			{\Huge\bfseries\LaTeX 使用简介} \\
			\rule[1em]{\linewidth}{0.5ex}
			
			\begin{tabular}{rr}
				作者 & XiaoCY \\
				版本 & 1.0 \\
				完成日期 & 2020-02-04 \\
				最后修改 & 2020-02-04
			\end{tabular}
		\end{flushright}
		
		\vspace{\stretch{3}}
		
		{\noindent 
			\textsl{Email}: 
			\href{mailto:chunyu2018@foxmail.com}
				{chunyu2018@foxmail.com}
		}
	\end{titlepage}

	\zihao{-4}				% 设置全文采用小四号字
	
	\pagenumbering{Roman}
	\tableofcontents		% 生成目录
	\clearpage
	
	\listoftables
	\clearpage
	
	\pagenumbering{arabic}
	\setcounter{page}{1}	% 正文重新开始于1
	
	\section{写在前面}
	\LaTeX 是一种区别于Word的排版软件,
	其格式由各种命令、环境控制,
	很容易做到内容与格式分离,
	这是它区别于Word的重要一点。
	这篇文章主要是本人学习 \LaTeX 的记录\cite{liuhaiyang},
	记录将常用的基本操作以便后期查阅,
	同时也为感兴趣的小伙伴们做一个粗略的介绍。
	
	本人学习时安装的发行版为 \TeX~Live,		% ~为空格
	对于不知道如何入门的小伙伴,
	我同样推荐这个发行版,
	它避免了很多繁琐的配置。
	编译器以\XeLaTeX 为主,
	编辑器可采用TeXstudio。
	
	\section{基础介绍}
	\subsection{文档组成}
	\LaTeX 文档的格式通常是以后缀.tex结尾的文本文件,
	为了正常使用中文和Unicode的特殊符号,
	务必将文本以UTF-8编码进行保存。
	
	\LaTeX 文档可分为导言区和正文区。
	导言区在命令\verb|\begin{document}|之前,
	通常对文档的性质做一些设置,
	也可以自定义一些命令。
	导言区之后为正文区,
	是文档的主要内容。
	
	在导言区内,我们首先需要声明文档类,
	本文采用了命令\verb|\documentclass{ctexart}|来声明C\TeX 文档类。
	此外,很多优秀的宏包可以辅助我们对文档格式进行控制,
	为了使用这些宏包,
	我们需要在导言区使用命令\verb|\usepackage{<宏包名>}| 。
	(如无特殊说明,本文的命令示意中以尖括号连同内容表示命令,
	使用时不需要添加尖括号。)
	
	在正文区内,我们只需要输入文档的正文即可。
	编辑器内部的各种换行不会引起排版后文档的换行,
	中 文 文 档 的 多 余 空 格 也 不 会 在 文 档 中 出 现。 \par
	文字换行可用双斜线\verb|\\| 实现,
	而换段则采用空行的形式,或者使用命令\verb|\par|。
	tex文件内的多个空行或多个\verb|\par|不会在正文引起多余的空行。
	换行和换段的区别是:
	当设置了段落首行缩进时,换行不会引入缩进。
	
	\subsection{使用符号}
	正文中绝大多数符号都可以由键盘直接输入,如@。
	但是有一些符号在 \LaTeX 中具有特殊的作用,
	因而不能直接在正文中使用。
	通常情况下我们可以在这些符号前面加上反斜线\textbackslash,
	但是也有个别例外。
	这些特殊符号的作用和使用方法如表\ref{tab:symbol} 所示。
	
	\begin{table}
		\centering
		\caption{特殊符号及其说明}
		\label{tab:symbol}
		\begin{tabular}{cll}
			\toprule
			符号 & 作用 & 输入方式\\
			\midrule
			\~{} & 不可打断的空格 & \verb|\~{}| \\
			\# & 用于宏定义 & \verb|\#| \\
			\$ & 数学模式 & \verb|\$| \\
			\% & 注释符 & \verb|\%| \\
			\^{} & 上标 & \verb|\^{}| \\
			\& & 用于表格对齐 & \verb|\&| \\
			\{\ \} & 用于分组 & \verb|\{ \}| \\
			\_ & 数学下标 & \verb|\_| \\
			\textbackslash & 宏命令和转义符 & \verb|\textbackslash| \\
			\bottomrule
		\end{tabular}
	\end{table}
	
	除此之外,符号-在 \LaTeX 正文中有很多用途:
	在数学模式下它是减号,如$3-1=2$;
	单独使用时它是英文连字符,如good-looking;
	两个连用(-\,-)时用来表示数字范围,如1--10;
	三个连用(-\,-\,-)时是破折号---比如这里。
	
	为了在正文中使用空格,可以采用反斜线和空格组合实现。
	此外,有一种称为幻影的神奇空格,
	它由命令\verb|\phantom{<内容>}|生成,
	空格的长度与\verb|<内容>|所占据的长度相同。
	
	\subsection{设置字体}
	字体具有五种不同的性质,
	在 \LaTeX 中一起决定了文字最终的输出效果。
	字号(font size)是指文字的大小,
	常常被独立出来,看作不同于字体的单独的性质;
	字体编码(font encoding)指字体包含的符号,
	一般情况下不直接进行设定。
	使用最多的是其他的三个性质:
	字体族(font family)、字体形状(font shape)、
	字体系列(font series)\footnote{通常指字体的粗细和宽度}。
	字体的这三个性质也常称为字体的坐标。
	
	\LaTeX 提供了各种命令来对字体进行修改。
	命令和效果如表 \ref{tab:fontcoord} 所示。
	
	\begin{table}
		\centering
		\caption{字体的坐标}
		\label{tab:fontcoord}
		\begin{tabular}{llll}
			\toprule
			字体族 & 带参数的命令 & 申明命令 & 效果 \\
			罗马 & \verb|\textrm{<text>}| & \verb|\rmfamily|
			    & \rmfamily Roman font family \\
			无衬线 & \verb|\textsf{<text>}| & \verb|\sffamily|
			    & \sffamily Sans serif font family \\
			打字机 & \verb|\texttt{<text>}| & \verb|\ttfamily|
			    & \ttfamily Typewriter font family \\
			\midrule
			字体形状 & 带参数的命令 & 申明命令 & 效果 \\
			直立 & \verb|\textup{<text>}| & \verb|\upshape|
			    & \upshape Upright shape \\
			意大利 & \verb|\textit{<text>}| & \verb|\itshape|
				& \itshape Italic shape \\
			倾斜& \verb|\textsl{<text>}| & \verb|\slshape|
				& \slshape Slanted shape \\
			小型大写 & \verb|\textsc{<text>}| & \verb|\scshape|
				& \scshape Small capitals shape \\
			\midrule
			字体系列 & 带参数的命令 & 申明命令 & 效果 \\
			中等 & \verb|\textmd{<text>}| & \verb|\mdseries|
				& \mdseries Medium series \\
			加宽加粗 & \verb|\textbf{<text>}| & \verb|\bfseries|
				& \bfseries Bold extended series \\
			\bottomrule
		\end{tabular}
	\end{table}

	中文字体没有英文字体那么复杂的成套的变体,
	各个字体之间一般都是独立的。
	因此,中文字体一般只用不同的字体族进行区分。
	ctex宏包和文档类(如ctexart)预定义了Windows常用的四种字体族:
	宋体、黑体、楷书、仿宋。
	为了方便使用,ctex宏包和文档类提供了简化命令:
	\verb|\songti| {\songti 宋体},
	\verb|\heiti| {\heiti 黑体},
	\verb|\kaishu| {\kaishu 楷书},
	\verb|\fangsong| {\fangsong 仿宋}。
	
	\verb|\zihao{<字号>}|可以用来修改字体的大小,
	例如\verb|\zihao{4}|将字号设置为四号,
	字号前添加负号如\verb|\zihao{-4}|则表示小四号。
	
	除了修改字体和字号外,
	ctex宏包和文档类还增加了各种文字强调的方式,
	命令和效果如表\ref{tab:fontemph} 所示。
	
	\begin{table}
		\centering
		\caption{文字强调}
		\label{tab:fontemph}
		\begin{tabular}{ll}
			\toprule
			命令 & 效果\\
			\midrule
			\verb|\CJKunderdot{<文字>}| & \CJKunderdot{强调文字} \\
			\verb|\CJKunderline{<文字>}| & \CJKunderline{强调文字} \\
			\verb|\CJKunderdblline{<文字>}| & \CJKunderdblline{强调文字} \\
			\verb|\CJKunderwave{<文字>}| & \CJKunderwave{强调文字} \\
			\verb|\CJKsout{<文字>}| & \CJKsout{强调文字} \\
			\verb|\CJKxout{<文字>}| & \CJKxout{强调文字} \\
			\bottomrule
		\end{tabular}
	\end{table}
	
	\subsection{定义命令}
	\LaTeX 中的命令又称为宏,它们都以反斜线开头,
	一般格式为:
	
	\begin{quote}	% 引用环境:增加前后间距和缩进
		\begin{tabular}{ll}
			无参数: & \verb|\command| \\
			有n个参数: & \verb|\command{<arg1>}{<arg2>}...{<argn>}| \\
			有可选参数: & \verb|\command[<opt>]{<arg1>}{<arg2>}...{<argn>}| 
		\end{tabular}
	\end{quote}

	例如,本文档在生成目录时就采用了命令\verb|\tableofcontents|。
	
	用户可以在导言区自定义命令,定义命令的基本格式为:
	
\begin{lstlisting}
\newcommand{<命令名>}[<参数个数>][<首参数默认值>]{<定义>}
\end{lstlisting}
	
	其中,命令名应当符合要求,以反斜线开头;
	当定义无参数命令时,
	参数个数和首参数默认值可连同中括号省略;
	当不给定默认值时,首参数默认值可连同中括号省略;
	当给定多个参数时,命令的定义中用井号加数字表示参数,
	如\verb|#2|表示第二个参数。
	
	例如我们可以定义一个命令来简化四元数乘法,
	按顺序输入两个四元数,
	输出四元数的乘法。
	定义为:
	
\begin{lstlisting}
\newcommand{\quadprod}[2]    % 可换行定义命令内容
    {\ensuremath{\mathfrak{#1} \otimes \mathfrak{#2}}}
\end{lstlisting}
	
	这样就定义了一个新命令\verb|\quadprod|。
	当使用\verb|\quadprod{q_b^a}{q_c^b}|时,
	编译后将输出\quadprod{q_b^a}{q_c^b}。
	
	除了用\verb|\newcommand|构造命令外,
	还可以使用\verb|\renewcommand|和\verb|\providecommand|。
	它们的用法相同。区别在于,
	\verb|\renewcommand|用于覆盖之前存在的同名命令;
	\verb|\providecommand|则会检测是否存在同名命令,
	若存在,则保留之前的用法,新定义不生效。
	
	\subsection{环境介绍}
	与命令相似,环境也可分为有参数的环境和无参数的环境。
	有参数环境的一般格式为:
	
\begin{lstlisting}
\begin{<环境名>}[<可选参数>]{<其他必要参数>} 
    <环境内容>
\end{<环境名>} 
\end{lstlisting}
	
	一个环境就是一个分组,它限定了一些命令的作用范围。
	除了使用环境,也可以用成对的花括号\{\}直接产生一个分组。
	
	我们给出一个分组的例子:
	为了将局部字体修改为黑体,
	只需要\verb|{\heiti 像这样}|构造分组,
	就能得到{\heiti 像这样}的局部黑体内容,
	而不会引起后面字体的改变。

	\section{编写文档}
	\subsection{页面设置}
	个人认为,页边距是使用 \LaTeX 编写文档时要考虑的第一个问题。
	因为如果最后修改页边距,
	一些所谓的盒子的内容有可能会超出边距,
	从而使文档需要重复调整。
	如果提前设置好边距,就能随时编译预览时发现问题。
	
	\LaTeX 对各种距离的定义较为复杂,
	为了能够快速修改,
	可以使用geometry宏包,
	并在导言区使用\verb|\geometry|命令对尺寸进行设置。
	
	如本文采用的设置为:
	
\begin{lstlisting}
\usepackage{geometry}
\geometry{ % 设置整体页面格式
	a4paper,
	left = 2.1cm,
	right = 2.1cm,
	bottom = 2.1cm,
	top = 2.5cm
}
\end{lstlisting}
	
	文档的第一页往往包含标题和作者等信息,
	也可以使用一个单独的封面。
	在标准文档类中,
	我们可以在正文中用\verb|\maketitle|生成标题页。
	该命令需要在导言区提前申明各种信息,如
	\verb|\title|、\verb|\author|等。
	
	除了使用预定义的标题页,
	我们也可以用titlepage环境自定义封面。
	该环境提供没有页码的单独一页,
	并使后面的内容页码从1开始计数。
	
	文档可使用\verb|\section|、\verb|\subsubsection|、
	\verb|\subsubsubsection|分别声明节、小节、小小节。
	\LaTeX 会自动进行编号。
	当需要使用到附录时,
	可使用\verb|\appendix|命令后,
	再分别使用\verb|\section|声明各级附录。
	附录格式的修改可使用appendix宏包,此处不具体说明。
	
	对于C\TeX 文档类
	\footnote{包括书籍ctexbook,报告ctexrep,文档ctexart。},
	可以使用\verb|\CTEXsetup|命令来设置各章节标题的格式:
	
\begin{lstlisting}
\CTEXsetup[<选项1>=<值1>,<选项2>=<值2>...]{<对象类型>}
\end{lstlisting}

	对于ctexart文档,常用的对象类型包括section、subsection
	和subsubsection等。
	常用的选项如表\ref{tab:ctexopt} 所示。
	
	\begin{table}
		\centering
		\caption{C\TeX 文档常用选项}
		\label{tab:ctexopt}
		\begin{tabular}{lll}
			\toprule
			选项 & 使用说明 & 值的示例 \\
			\midrule
			name & {<前名>,<后名>} & \verb|{第,节}| \\
			number & 设置编号的格式 & \verb|{\chinese{section}}| \\
			format & 章节名和章节标题的全局格式 & \verb|{\bfseries}| \\
			nameformat & 章节名和编号的格式 & 同format \\
			numberformat & 仅控制编号格式 & 同format \\
			titleformat & 仅控制章节标题格式 & 同format\\
			aftername & 章节名与标题之间的内容 & \verb|{\hspace{2ex}}| \\
			beforeskip & 章节标题前的段间距 & \verb|{\vspace{2em}}| \\
			afterskip & 章节标题后的段间距 & 同beforeskip \\
			indent & 章节标题的缩进长度 & 同aftername \\
			\bottomrule
		\end{tabular}
	\end{table}
	
	\subsection{添加目录}
	目录是最基本的自动化工具。
	\LaTeX 会自动收集章节命令所定义的各章节标题,
	用命令\verb|\tableofcontents|即可输出。
	类似地,命令\verb|\listoffigures|和\verb|\listoftables|
	会分别收集figure和table中\verb|\caption|命令的图表标题,
	产生图表的目录。
	
	目录部分的页码常用大写的罗马数字表示,
	这可以使用命令\verb|\pagenumbering{<格式>}|进行修改,
	其中格式允许的值如表\ref{tab:pagenum} 所示。
	
	在目录部分结束之后,
	常换页进入正文部分,
	并将页码计数重置为1,
	格式设置为阿拉伯数字,
	可采用以下命令:
	
\begin{lstlisting}
\clearpage					% 换页
\pagenumbering{arabic}		% 设置页码格式
\setcounter{page}{1}		% 正文重新开始于1
\end{lstlisting}
	
	\begin{table}
		\centering
		\caption{页码数字格式}
		\label{tab:pagenum}
		\begin{tabular}{ll}
			\toprule
			格式 & 说明 \\
			\midrule
			arabic & 阿拉伯数字 \\
			roman & 小写的罗马数字 \\
			Roman & 大写的罗马数字 \\
			alph & 小写的字符形式 \\
			Alph & 大写的字符形式 \\
			\bottomrule
		\end{tabular}
	\end{table}
	
	\subsection{段落格式}
	在C\TeX 文档类中已经预设了正确的首行缩进,
	如果需要在某一段临时取消缩进,
	可以使用\verb|\noindent|命令;
	相反,为了在本来没有缩进的环境中临时设置缩进,
	可使用\verb|\indent|命令。
	这两个命令的作用范围均为一个段落,
	因而不需要再进行分组加以限定。
	
	除了段落的首行缩进,另一个关于分段的重要参数是段与段之间的垂直距离。
	这个距离由变量\verb|\parskip|控制,可以采用命令
	\verb|\setlength{\parskip}{<length>}|进行修改。
	这时应当注意,该命令会对后续的段落均产生影响。
	因此当只修改某些段落的段间距时,
	需要用花括号限制\verb|\setlength|的作用范围。
	
	在一些情况下,我们需要在文档内插入一段引文。
	quote环境常用来引用小段文字,
	该环境下没有首行缩进,
	且左右边距比正常文本稍微大一些,
	且增加了环境前后段的间距。
	比如下面给出一个使用这种环境的例子。
	
	\begin{quote}
		学而时习之,不亦说乎?
	\end{quote}
	
	而对于大段文字的引用,
	通常需要引入首行缩进,
	这时可采用quotation环境,
	如下面一段文字。
	
	\begin{quotation}
		小车正穿行在落基山脉蜿蜒曲折的盘山公路上。
		克里朵夫·李维静静地望着窗外,
		发现每当车子即将行驶到无路的关头,
		路边都会出现一块交通指示牌:
		“前方转弯!”或“注意!急转弯”。
		而拐过每一道弯之后,
		前方照例又是一片柳暗花明、豁然开朗。
		
		山路弯弯、峰回路转,
		“前方转弯”几个大字一次次地冲击着他的眼球,
		也渐渐叩开了他的心扉:
		原来,不是路已到了尽头,而是该转弯了。
		路在脚下,更在心中,心随路转,心路常宽。
		学会转弯也是人生的智慧,
		因为挫折往往是转折,危机同时是转机。
	\end{quotation}
	
	\subsection{定理环境}
	定理环境实际上是一类环境,在使用前需要在导言区进行定义:

\begin{lstlisting}
\newtheorem{thm}{定理}
\end{lstlisting}
	
	这样我们就得到了一个名为thm的定理类环境,
	它在使用时会自动产生形如“定理1”的提示。
	用同样的方法我们还可以定义引理、公理等。
	有时候我们希望定理的编号包含章节号,
	可在定义时增加参数即可:

\begin{lstlisting}
\newtheorem{thm}{定理}[section]
\end{lstlisting}
	
	定理环境还可以有一个可选参数,即定理的名字。
	下面给出这个环境的使用示例:
	
	\begin{thm}[勾股定理]
		直角三角形斜边的平方等于两直角边的平方和。
	\end{thm}	

	ntheorem宏包扩充了定理类环境的格式,
	在导言区使用\verb|\theoremstyle{<格式>}|可以方便地选择格式。
	可用的预定义格式如表\ref{tab:thmstyle} 所示。
	
	\begin{table}
		\centering
		\caption{定理类环境格式}
		\label{tab:thmstyle}
		\begin{tabular}{ll}
			\toprule
			格式 & 说明 \\
			\midrule
			plain & 默认格式 \\
			break & 定理头换行 \\
			marginbreak & 编号在页边,定理头换行 \\
			changebreak & 定理头编号在前文字在后,换行 \\
			change & 定理头编号在前文字在后,不换行 \\
			margin & 编号在页边,定理头不换行 \\
			nonumberplain & 同plain格式,没有编号 \\
			nonumberbreak & 同break格式,没有编号 \\
			empty & 没有编号和定理名,只输出可选参数 \\
			\bottomrule
		\end{tabular}
	\end{table}
	
	该宏包具有很多设置命令和辅助功能,
	比如在使用该宏包时添加\verb|[thmmarks]|选项后,
	可以使用\verb|\theoremsymbol|命令在定理类环境末尾添加符号,
	这对定义证明环境表示证毕符号非常有用:
	
\begin{lstlisting}
% 导言区
\usepackage[thmmarks]{ntheorem}
{	% 利用分组使设置只对该分组内的定理类有效
	\theoremstyle{nonumberplain}
	\theoremheaderfont{\bfseries}
	\theorembodyfont{\normalfont}
	\theoremsymbol{\ensuremath{\Box}}
	\newtheorem{proof}{证明}
}
\end{lstlisting}
	
	这样,使用该定理类将会出现下面的效果:
	
	\begin{proof}
		当$x>0$时,将$\mathrm{e}^x$进行泰勒展开,有
		
		\[
			\mathrm{e}^x = \sum_{k=0}^{+\infty} \frac{x^k}{k!} 
			>\sum_{k=0}^{n} \frac{x^k}{k!}=P_n(x)
		\]
		
		证毕。
	\end{proof}
	
	\subsection{添加列表}
	\LaTeX 标准文档类提供了三种列表环境:
	带编号的enumerate环境、
	不编号的itemize环境
	和使用关键字的description环境。
	在列表环境内部使用\verb|\item|命令开始一个新的列表项,
	它可以带一个可选参数表示手动编号或关键字。
	
	这三种列表环境可以嵌套使用(最多四层),
	\LaTeX 会自动处理不同层次之间的缩进编号,
	例如:
	
	\begin{enumerate}
		\item 使用enumerate环境直接用\verb|\item|产生一个列表项
		\item[1b.] 使用\verb|\item[1b.]|测试手动编号
		\item 这里恢复正常的使用方法,并引入一次itemize嵌套
		\begin{itemize}
			\item 嵌套的第一层默认编号
			\item[\dag] 使用\verb|\item[\dag]|修改符号
			\item 恢复到默认设置
		\end{itemize}
	\end{enumerate}
		
	\subsection{使用图表}
	%	表格环境table和图片环境figure均为浮动环境。
	%	与一般的环境不同,
	%	浮动环境可以根据文档的内容改变位置。
	%	在使用浮动环境时可利用可选参数h、t、b、p的组合来指明允许的浮动位置。
	%	其中:h表示代码中与前后文关系不变的位置;
	%	t表示当前页面的顶端;
	%	b表示当前页面的底端;
	%	p表示下一页。


	\subsection{数学公式}
	\AmS 相关宏包提供了很好的数学支持,
	这些宏包包括:amsmath、amsfonts、amssymb。
	
%	公式编号往往需要带上章节号,
%	在导言区添加\verb|\numberwithin{equation}{section}|来实现。
	
	\subsection{插入代码}
	\LaTeX 输入特殊符号时不太便利,
	然而有时候我们必须经常性地使用特殊符号,
	例如在排版计算机程序源代码的时候。
	此时需要使用抄录功能。
	
	使用命令\verb|\verb<符号><抄录内容><符号>|
	可以将抄录内容原封不动地输出到正文中,
	两端的符号可以任意给定,但是要确保相同。
	例如\verb~\verb|\TeX|~采用“|”作为边界。
	
	在插入大段代码时,可以使用listings宏包提供的lstlisting环境。
	但是直接使用往往得不到很好的效果,
	通常我们需要在导言区使用\verb|\lstset|进行设置。
	例如本文的设置如下:

\begin{lstlisting}
% 导言区
\usepackage{listings}
\usepackage{listings}
\lstset{ % 代码环境整体设置
	basicstyle = \ttfamily,
	keywordstyle = \bfseries,
	commentstyle = \rmfamily\upshape,
	stringstyle = \ttfamily\slshape,
	tabsize = 4,
	backgroundcolor = \color{lightgray},	% 需先引入颜色宏包
	frame = single,
	language = TeX							% 设置默认语言
}
\end{lstlisting}
	
	需要说明的是,lstlisting环境对缩进是敏感的。
	这就是说,从\verb|\begin{lstlisting}|开始的缩进都将进入正文。
	该环境还可以使用可选参数,
	对环境进行临时修改。
	为了使代码根据不同语言实现高亮,
	可以在使用该环境使添加可选参数language进行设置,
	例如上面的例子就引入了参数\verb|[language=TeX]|。
	
	\subsection{插入引用}
	最简单的引用便是使用\verb|\footnote{<脚注内容>}|产生脚注,
	例如这个位置\footnote{这是一个脚注的例子}。
	脚注是自动编号的,也可以使用可选参数修改编号,
	但是这不改变原来脚注的编号。
	
	在例如表格等环境中,\verb|\footnote|命令往往不能直接使用。
	这时我们可以把脚注标记和脚注内容分开\footnotemark,
	这两条命令分别为
	\verb|\footnotemark|和\verb|\footnotetext{<脚注内容>}|。
	\footnotetext{这又是一个脚注}
	
	\LaTeX 默认的脚注格式不太好看,
	我们可以在导言区进行以下设置进行优化。
	
\begin{lstlisting}
\usepackage[perpage]{footmisc}		% 脚注每页清零
\usepackage{pifont}					% 优化带圈的数字
\renewcommand{\thefootnote}{\ding{\numexpr171+\value{footnote}}}
\end{lstlisting}
	
	交叉引用可以通过一个符号标签引用文档中某个对象的
	编号、页码、或标题等信息,
	而不必知道这个对象具体在什么位置。
	我们需要两个步骤来实现交叉引用:
	定义标签和引用标签。
	
	定义标签是在合适的位置给一个带参数的对象添加标签,
	命令为\verb|\label{<标签>}|。
	标签的名字最好是简洁而有用的名字,
	不能包含特殊字符。
	标签的位置常在\verb|\section|命令之后;
	或table和figure环境中\verb|\caption|命令之后;
	也可以在公式之后。
	
	创建标签后,可分别使用
	\verb|\ref|、\verb|\pageref|、\verb|\nameref|
	引用标签对象的计数、所在页码和名称。
	对于公式的引用,通常会给编码加上英文括号,
	好在 \AmS 的宏包在支持数学的基础上
	增加了命令\verb|\eqref|专门用于公式引用,
	所以公式的引用会略有不同。
	
	例如,我们使用下面的代码进行交叉引用的测试:

\begin{lstlisting}
% 前面已使用 \label{tab:ctexopt} 定义标签
表 \ref{tab:ctexopt} 在第 \pageref{tab:ctexopt} 页,
名字是 \nameref{tab:ctexopt}。
\end{lstlisting}
	
	输出为:表 \ref{tab:ctexopt} 在第 \pageref{tab:ctexopt} 页,
	名字是 \nameref{tab:ctexopt}。
	
	细心的朋友对比后会发现,
	\verb|\tableofcontents|生成的目录不是超级链接,
	而且编译生成的pdf文档中不会出现标签。
	为了解决这个问题,
	可以使用hyperref宏包。
	据了解\cite{liuhaiyang},
	hyperref宏包可以算是 \LaTeX 中最为复杂的宏包之一。
	它提供了大量的选项和命令,
	能够完成各种设置和功能。
	但限于本人的水平,在此不做说明。
	需要指出的是,在中文文档中,
	直接使用\verb|\usepackage{hyoerref}|可能会出现乱码,
	解决这个问题的方式只需要在申明文档类时加上可选项即可,
	如下
	
\begin{lstlisting}
% 导言区
\documentclass[hyperref,UTF8]{ctexart}
\end{lstlisting}
	
	同时,该宏包还提供了命令
	\verb|\url{<URL>}|和\verb|\href{<URL>}{文字}|
	产生超级链接。
	比如,这里我可以班门弄斧一下
	\href{https://ichunyu.github.io}{我的博客}。
	
	除了上面这些引用,
	科技类文档中最重要的引用就是文献引用了。
	这类引用首先需要一个数据库文件,
	其通常以bib为后缀。
	该文件由多项参考文献构成,
	这里给出其中的一个例子:
	
\begin{lstlisting}
@article{canuto_embedded_2018,
title = {Embedded model control: {Reconciling} modern control theory and error-based control design},
volume = {16},
issn = {2198-0942},
shorttitle = {Embedded model control},
url = {https://doi.org/10.1007/s11768-018-8130-1},
doi = {10.1007/s11768-018-8130-1},
language = {en},
number = {4},
urldate = {2019-07-16},
journal = {Control Theory and Technology},
author = {Canuto, Enrico and Novara, Carlo and Colangelo, Luigi},
month = nov,
year = {2018},
keywords = {embedded model control, disturbance rejection, error loop, error-based design, Modern control theory},
pages = {261--283}
}
\end{lstlisting}
	
	通常情况下bib文件不需要我们手动编写,
	文献检索网站和各种文献管理软件一般都支持这种BibTeX格式的导出。
	
	文献的引用需要以下三个步骤:
	
	\begin{enumerate}
		\item 使用\verb|\bibliographystyle|命令设定参考文献的格式,
		这通常在导言区完成。基本的文献格式有:
		plain --- 按作者、日期、标题排序;
		unsrt --- 不排序但保持引用次序;
		alpha --- 使用一种三字母缩写的方式编号并按作者排序;
		abbrv --- 与plain基本相同,但是定义了一些缩写。
		
		\item 在正文中使用\verb|\cite|命令引用所需要的文献,
		当引用多个文献时用逗号隔开。
		
		\item 使用\verb|\bibliography|命令指明要使用的文献数据库,
		即包含文献信息的bib文件。
		同时, \LaTeX 会在这个命令的位置插入参考文献列表。
	\end{enumerate}

	例如,使用以下命令进行文献引用:

\begin{lstlisting}
Canuto教授提出了一种模型嵌入控制\cite{canuto_embedded_2018}。
\end{lstlisting}
	
	其效果为:Canuto教授提出了一种模型嵌入控制\cite{canuto_embedded_2018}。
	参考文献列表见本文末尾。
	
	默认的文献样式不能满足所有出版社的要求,
	这时我们可以使用natbib宏包进行修改。
	该宏包同时定义三种专用的格式:
	plainnat、abbrvnat、unstrnat。
	
	natbib宏包提供了\verb|\setcitestyle|命令来设置引用命令的输出格式,
	在参数中可以设置以下选项:
	
	\begin{itemize}
		\item 选择引用模式:
		authoryear代表作者年代模式,如[Canuto,2018];
		numbers表示数字序号模式,如[3];
		super表示数字上标模式,如${}^{[64]}$。
		
		\item 括号:
		round圆括号,
		square方括号,
		或是用\verb|open={<左括号>}|和\verb|close={<右括号>}|
		分别进行设置。
		
		\item 多个引用之间的标点:
		semicolon分号,comma逗号,
		或是用\verb|citesep={<符号>}|进行设置。
		
		\item 作者与年代之间的符号:\verb|aysep={<符号>}|。
		
		\item 同一作者的几个年代间的符号\verb|yysep={<符号>}|。
		
		\item 在引用命令可选参数的说明文字前的符号\verb|notesep={<符号>}|。
	\end{itemize}

	简单起见,我们可以在使用宏包时添加选项来快速修改格式,如下:
	
\begin{lstlisting}
\usepackage[super,square]{natbib}
\end{lstlisting}
	
	\subsection{使用颜色}
	
	
	\section{使用模板}
	
	
	\bibliography{ref}
\end{document}
