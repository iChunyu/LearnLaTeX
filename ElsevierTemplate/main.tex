\section{准备工作}

本文假设小伙伴已经正确安装了 \TeX~Live 并对个人爱好的编辑器进行了相应的配置。安装与配置可以参考我博客里的\href{https://ichunyu.github.io/categories/latex/}{系列文章}。Elsevier 的模板已被该发行版收录,且本文所需的其他所有宏包均无需再手动下载。


\LaTeX 在编译过程中会产生大量的临时文件,通常由独立的文件夹来管理不同文档。对于期刊文章这类小型文档,个人推荐的文件结构由以下5部分构成:

\begin{itemize}
    \item 顶层设置文件

        类似与各类程序的主函数,建议使用独立的 \verb|.tex| 文件管理文件结构。在该顶层文件中对期刊模板进行设置,例如文章标题、作者、地址等信息。此外,顶层文件还对修订过程中临时使用的宏包进行设置,并利用 \verb|\input| 命令引入正文所需的宏包以及正文内容、管理致谢、参考文献等章节的基本顺序。这样做的好处在于:不同期刊提供了不同的模板,当需要更换不同期刊模板时,只需要创建新的顶层文件并进行相应修改即可。本文的顶层文件为 \verb|ElsevierPaper.tex|。

    \item 正文内容文件

        使用独立的文件编写正文内容,有利于将内容与格式分离。文章的所有内容(从引言开始)都汇总在单个文件中,在后续需要修订时,只需要编辑正文文件即可。同时,如果手动进行版本控制,只需要对正文文件进行备份,而不需要备份整个文件夹。本文的正文内容文件为 \verb|main.tex|。

    \item 宏包设置文件

        使用独立的文件单独管理正文所需的宏包,并进行相应设置。这样的好处是在更换模板时可以直接引入之前的配置,而不必对正文内容重新进行配置。本文的宏包设置文件为 \verb|packages.tex|。

    \item 参考文献数据库

        正文所应用的参考文献及其相关信息汇总在独立的 \verb|.bib| 数据库文件中,该文件通常可以通过外部文献管理软件导出。本文的参考文件数据库为 \verb|references.bib|。

    \item 图片文件夹

        正文可能会插入多个图片,可汇总在单独的文件夹内。本文的图片文件夹为 \verb|figures/|。
\end{itemize}


本文尽可能采用干净的代码风格,旨在帮助读者将本文与其源码相结合,更快地熟悉 \LaTeX 文档编写规则。同时,读者可直接基于本模板进行文章的编写,完成后依据顶层文件的注释删除汉化设置即可。


\TeX~Live 收录了 \verb|latexmk| 脚本,可用于自动编译文档。当使用了 \verb|ctex| 宏包且正文包含中文时(注释除外),应使用 \verb|latexmk -xelatex ElsevierPaper.tex| 进行编译(注意替换顶层文件的名字);当文章内容为全英文时,通常应当将命令改为 \verb|latexmk -pdf ElsevierPaper.tex|。特别地,如果在上述命令中添加 \verb|-pvc| 选项,例如 \verb|latexmk -pdf -pvc ElsevierPaper.tex|,该脚本将会在后台运行并监测文件改动。当文件保存时会自动编译。如此做,如果保存的文件包含错误命令,后台的 \verb|latexmk| 脚本会暂停,需要在命令行的提示处输入 \verb|R| 使其重新启动,或输入 \verb|X| 退出脚本。




\section{标题页设置}


\section{编写正文}

\subsection{插图}
\subsection{表格}
\subsection{公式}
\subsection{引用}

\section{修订文档}

\section{高级技巧}


