% =============== 导言区 ===============
\documentclass{springdoc}

\title{我的 \LaTeX 模板}
\author{XiaoCY}
\version{1.1}
\finishdate{2020-02-02}
\lastupdate{2020-02-02}
\email{chunyu2018@foxmail.com}

% =============== 正文区 ===============
\begin{document}
	
	\maketitle
	
	\pagenumbering{Roman}
	
	\begin{vertab}	% 版本控制
		1 & 2020-02-13 & 创建文档 \\
	\end{vertab}

	\begin{abbrtab}	% 简称说明
		1 & PID & 比例-积分-微分控制器 \\
	\end{abbrtab}

	\begin{symtab}	% 符号说明
		1 & $\tilde{\mathfrak{q}}$ & 误差四元数 \\
	\end{symtab}

	\clearpage
	
	\tableofcontents		% 生成目录
	\clearpage
	
	\listoffigures
	\clearpage
	
	\listoftables
	\clearpage

	\pagenumbering{arabic}
	\setcounter{page}{1}	% 正文重新开始于1
	
	\section{使用说明}
	按照排版的顺序,简要说明本模板的使用方法。
	首先在导言区申明文档类为springdoc模板;
	在导言区分别给出标题、作者、时间、邮箱等信息。
	
	在正文区,可使用\verb|\maketitle|生成封面,
	该命令已由模板文件重载。
	
	模板定义了扉页的相关说明性表格,
	分别可由vertab、abbrtab、symtab环境直接给定内容,
	也可以不与采用。
	
	tex文件中\verb|\tableofcontents|用于生成目录,
	\verb|\listoffigures|、\verb|\listoftables|分别生成图表目录,
	可根据需求取舍。
	
	\subsection{正文编辑}
	
	正文编辑规则与 \LaTeX 基本规则相同,
	可参考
	\href{https:github.com/iChunyu/LearnLaTeX}{仓库}
	中的相关简介。
	
	本模板对代码环境进行了调整,
	默认代码语言为MATLAB,
	需要更改时可在环境中加入可选参数,
	如\verb|\begin{lstlisting}[language = C]|。
	
	
\begin{lstlisting}
% decode_txt decodes hex text file to normalized decimal data
% All data is decoded as a complement
% Usage: NormData = decode_txt(FileName,ByteFormat,DataIndex)
%     FileName   --- Full name of data file
%     ByteFormat --- Byte format of data packet
%     DataIndex  --- The index of the data in pacaket
% Example:
%     FileName = 'ExpData.dat';
%     ByteFormat = [  4 ...   % 1:   Header
%                 3 3 3  ...  % 2-4: Three acceleration data X/Y/Z
%                 2 2 2 ];    % 5-7: Three translation data X/Y/Z
%         E.g. ByteFormat(5) = 2, according to the comments:
%                 5 means the 5th data in pacaket, which is X translation;
%                 2 means the data takes 2 bytes.
%     DataIndex = [ 2 7 ];    % Extract 2nd and 13th data in pacaket
%     Call decode_txt, output is X acceleration and Z translation.
% Notice that: Output data is normalized!

% XiaoCY 2019-06-14

%% Main
function NormData = decode_txt(FileName,ByteFormat,DataIndex)
	FileID = fopen(FileName,'r');
	RawData = textscan(FileID,repmat('%2s',1,sum(ByteFormat)), ...
		'TextType','string');
	fclose(FileID);
	
	for k = 1:length(RawData)
		RawData{k} = strtrim(RawData{k});
	end
	RawData = [RawData{:}];
	
	[NPoint,~] = size(RawData);
	NIndex = length(DataIndex);
	DecData = zeros(NPoint,NIndex);
	NormData = zeros(NPoint,NIndex);
	
	for k = 1:NIndex
		StartByte = sum(ByteFormat(1:DataIndex(k)-1))+1;
		EndByte = StartByte+ByteFormat(DataIndex(k))-1;
		MSB = 2^(ByteFormat(DataIndex(k))*8-1);
		for m = 1:NPoint
		DecData(m,k) = hex2dec([RawData{m,StartByte:EndByte}]);
	end
	IndexN = DecData(:,k)>MSB;          % Negative data index
	IndexP = ~IndexN;                   % Positive data index       
	NormData(IndexP,k) = DecData(IndexP,k)/MSB;
	NormData(IndexN,k) = (DecData(IndexN,k)-MSB*2)/MSB;
end
\end{lstlisting}
	
	\bibliography{ref}
	
	\appendix
	
	\section{一级附录}
	\subsection{二级附录}

\end{document}
